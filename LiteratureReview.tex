\documentclass[report]{IEEEtran}
\usepackage{blindtext}

%\title{The Emerging Technology of Blockchains}
%\author{Ian Burke - G00307742 }
%\date{\today}

\begin{document}
\title{Economic Implications of the Blockchain Technology}
\author{Ian Burke % <-this % stops a space
\thanks{I. Burke is a student studying
Software Development, Galway-Mayo Institute of Technology, Old Dublin Rd, Galway, Ireland e-mail: g00307742@gmit.ie}% <-this % stops a space
}

\maketitle

\begin{abstract}
Text for abstract
\end{abstract}


\section{Introduction}
Blockchain is the driving force behind the popular cryptocurrency Bitcoin. A blockchain is a digital ledger that is distributed across a network of nodes, each node acquiring an updated copy of the ledger. The ledger contains a list of all transactions that have been executed on the network. All transactions are verified by consensus of the network with a cryptographic audit trail maintained and validated by multiple nodes. The distributed ledger is immutable. It can be updated by adding new sets of transactions but it cannot be changed, erased or mutated any further. Every transaction can be viewed on the public ledger, but the identity of participating parties behind every transaction is kept hidden. The blockchain uses a peer-to-peer network [1] meaning there is no central authority in control of the system. Therefore, creating a decentalised environment. Blockchain allows parties to transact directly to each other without the involvement of a trusted intermediary or third party i.e. there is no middleman needed. This leads to faster and cost effective transactions. The system is secured using asymmetric cryptographic algorithms and protocols.
\\\\ Since the emergence of Bitcoin, a lot of attention began to circulate around blockchain and the capabilities the technology can provide beyond cryptocurrencies. Blockchain is capable of disrupting industries like business and finance to state governance and giving empowerment to people in a decentralised fashion [4]. Blockchain applications such as distributed cloud storage, digital identity, digital voting and decentralised notary are currently being developed [7]. The introduction of smart contracts as another blockchain component allow us to prove ownership of a digital property or any asset of value. Physical assets such as a house, car or smartphone or non-physical assets such as company shares can be implemented with smart contracts. Since the blockchain is a distributed ledger, it can also store a proof of existence of legal documents, health records, personal identity, notary i.e. document certification and private securities. A sense of anonymity and privacy can be achieved by storing the fingerprint of digital assets. 
\\\\ The paper is structured as follows: In section II, we will examine how blockchain works and the potential it can pose with smart contracts. In section III, we will discuss economic implications of blockchain and in the final section we will present our conclusion.

\subsection{subsection}

\section{How does it work}
E-commerce on the internet relies heavily on financial institutions as trusted intermediaries or third parties to process and mediate electronic transactions. This system is not ideal with the chance of fraudulent activity and high transaction costs. Security of the system is based on trust [1]. 

Blockchain was first introduced with Bitcoin to solve the double-spending problem [1],[5]. A blockchain uses a peer-to-peer network which is comprised of interconnected nodes that share resources among each other without the use of a centralized administrator that controls the system i.e. decentralised. Instead of using a middleman to carry out a transaction, the Bitcoin blockchain uses cryptography to verify and validate transactions [1]. When one party wants to send Bitcoin to another party, each transaction is signed with a digital signature, to ensure the authentication of both parties. The sender owns a private key and sends out a public key which the receiver uses to verify the owner. Each transaction is broadcasted to all nodes across the network and verified by a consensus i.e. majority of nodes agree on a transaction. Verified transactions are placed into blocks with each block containing a set of transactions. A set of transactions in one block are considered to have occurred at the same time [6]. Once a block is validated through cryptographic proof-of-work, it is then timestamped and giving a hash i.e. a reference of the previous block. Each validated block is linked to each other in a linear, chronological order, hence the name blockchain. A new block is added roughly every ten minutes after a node solves the mathematical problem [6]. This helps reduce the probability that a block will be generated more than once at a given time. A consensus is made about the ordering of blocks. Nodes that participate in generating proof-of-work are called miners. The miners use their CPU power to solve these mathematical puzzles. Each miner competes with each other to complete the proof-of-work and thus add a newly mined block to the chain in order to be awarded Bitcoin as an incentive [1].  
\subsection{Peer-to-Peer Network}
A distributed network  Users can establish a connection point on the blockchain via any node on the network. Users engage with the blockchain using public and private keys. A private key is used to verify their own transaction while the public key is used to broadcast the transaction to the network. Each node collects and validates a transaction. Once a transaction is validated by nodes i.e. miners, it is then timestamped and placed in a block. The block is verified by nodes to see that all transactions in that block are valid before it can be accepted and placed on the chain. A block containing invalid transactions will then be discarded by the nodes. Lets say nodes broadcast different variations of a block at the same time
\subsection{Security}
The Bitcoin Blockchain is secured using cryptographic algorithms and protocols such as hashing and asymmetric private/public keys. For instance, a private key is used to create a digital signature for each transaction a user broadcasts to the network. The digital signature is used to authenticate that the transaction belongs to the user and also prevents data corruption and manipulation from attackers. The private key is anonymous while the public key is used as proof that the transaction belongs to the particular user. 
\subsection{Distributed Data Storage}
\subsection{Decentralization}
\section{Conclusion}

\begin{thebibliography}{1}

\bibitem{1}
Satoshi Nakamoto, \emph{Bitcoin: A Peer-to-Peer Electronic Cash System}, https://bitcoin.org/bitcoin.pdf, 2008.
\bibitem{2}
https://cacm.acm.org/magazines/2016/11/209132-blockchain-beyond-bitcoin/fulltext
\bibitem{3}
Decentralized Blockchain Technology and Rise of Lex Cryptographia
\bibitem{4}
Blockchain: Blueprint for a new economy
\bibitem{5}
https://www.bitcoin.com/info/what-is-bitcoin-double-spending
\bibitem{6}
http://scet.berkeley.edu/wp-content/uploads/AIR-2016-Blockchain.pdf
\bibitem{7}
https://www.huffingtonpost.com/ameer-rosic-/5-blockchain-applications_b_13279010.html

\end{thebibliography}

\end{document}
