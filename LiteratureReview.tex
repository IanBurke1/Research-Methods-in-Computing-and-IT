\documentclass[report]{IEEEtran}
\usepackage{blindtext}

%\title{The Emerging Technology of Blockchains}
%\author{Ian Burke - G00307742 }
%\date{\today}

\begin{document}
\title{Economic Implications of the Blockchain Technology}
\author{Ian Burke % <-this % stops a space
\thanks{I. Burke is a student studying
Software Development, Galway-Mayo Institute of Technology, Old Dublin Rd, Galway, Ireland e-mail: g00307742@gmit.ie}% <-this % stops a space
}

\maketitle

\begin{abstract}
Text for abstract
\end{abstract}


\section{Introduction}
In 2009, the world was first introduced to Bitcoin, a decentralized digital cash system where parties can complete transactions directly with one another without a middleman or an intermediary. No one knows who created Bitcoin, their true identity is unknown although the person or group of persons have used an online alias, Satoshi Nakamoto. The cryptocurrency is now used worldwide. The driving force behind Bitcoin is Blockchain, a distributed ledger that contains a list of completed transactions and stores them in blocks which are validated by nodes on the network. Each validated block is timestamped and chained together to increase security and to eliminate fraud, hence the name "Blockchain". This technology is revolutionising not just for cryptocurrencies like Bitcoin but for other future applications in industries such as finance, health care, social and the government sector. Smart contracts is also another popular topic associated with Blockchains. These digital contracts are self executing which are a which are used in the blockchain to enable people to exchange money, stocks, property and other potential assets. A Blockchain offers a sense of security and privacy to users thanks to its cryptographic algorithms. Another key feature of blockchains are smart contracts.  In this review, I will discuss the implications of the Blockchain technology on the economy and how it can solve real world problems.
\subsection{subsection}

\section{Inside the Blockchain}
In simple terms, a Blockchain is basically a digital ledger that records and logs every completed transaction. 
\subsection{Peer-to-Peer Network}
A distributed network comprised of interconnected nodes that share resources among each other without the use of a centralized administrator that controls the system. Users can establish a connection point on the blockchain via any node on the network. Users engage with the blockchain using public and private keys. A private key is used to verify their own transaction while the public key is used to broadcast the transaction to the network. Each node collects and validates a transaction. Once a transaction is validated by nodes i.e. miners, it is then timestamped and placed in a block. The block is verified by nodes to see that all transactions in that block are valid before it can be accepted and placed on the chain. A block containing invalid transactions will then be discarded by the nodes. 
\subsection{Cryptography}
\subsection{Distributed Data Storage}
\subsection{Decentralization}
\section{Conclusion}

\begin{thebibliography}{1}

\bibitem{Bitcoin}
Satoshi Nakamoto, \emph{Bitcoin: A Peer-to-Peer Electronic Cash System}, https://bitcoin.org/bitcoin.pdf, 2008.

\end{thebibliography}

\end{document}
