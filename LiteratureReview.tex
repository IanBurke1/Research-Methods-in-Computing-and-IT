\documentclass[report]{IEEEtran}
\usepackage{blindtext}

%\title{The Emerging Technology of Blockchains}
%\author{Ian Burke - G00307742 }
%\date{\today}

\begin{document}
\title{Economic Implications of the Blockchain Technology}
\author{Ian Burke % <-this % stops a space
\thanks{I. Burke is a student studying
Software Development, Galway-Mayo Institute of Technology, Old Dublin Rd, Galway, Ireland e-mail: g00307742@gmit.ie}% <-this % stops a space
}

\maketitle

\begin{abstract}
Text for abstract
\end{abstract}


\section{Introduction}
The term "revolutionize" is been thrown around in relation to blockchain technology. A technology capable of disputing industries like business and finance to state governance and giving empowerment to people in a decentralised fashion [4]. \\\\

A blockchain is a digital ledger that is distributed across a network of nodes, each node acquiring an updated copy of the ledger. The ledger contains a list of all transactions that have been executed on the network. Majority of nodes verifies each transaction by forming a consensus i.e. making decisions in the best interest of the network. The distributed ledger is immutable. It can be updated by adding new sets of transactions but it cannot be changed, erased or mutated any further. Every transaction can be viewed on the public ledger, but the identity of participating parties behind every transaction is kept anonymous. The blockchain uses a peer-to-peer network [1] meaning there is no central authority in control of the system. Therefore, creating a decentalised environment. Blockchain allows parties to transact directly to each other without the involvement of a trusted intermediary or third party like a bank. There is no middleman needed. This leads to faster and cost effective transactions. The system is secured using cryptographic algorithms and protocols.

\\\\ Since the emergence of Bitcoin, a lot of attention began to circulate around blockchain and the capabilities the technology can provide beyond cryptocurrencies.  

\\\\ In section II, we will examine what a blockchain is and how the technology works. In section III, we will discuss future blockchain applications and how blockchain will change society, business and finance. At the end of the review, we will present our conclusion in the final section.

\subsection{subsection}

\section{Inside the Blockchain}
In simple terms, a blockchain is basically a digital ledger that records and logs every completed transaction. 
\subsection{Peer-to-Peer Network}
A distributed network comprised of interconnected nodes that share resources among each other without the use of a centralized administrator that controls the system. Users can establish a connection point on the blockchain via any node on the network. Users engage with the blockchain using public and private keys. A private key is used to verify their own transaction while the public key is used to broadcast the transaction to the network. Each node collects and validates a transaction. Once a transaction is validated by nodes i.e. miners, it is then timestamped and placed in a block. The block is verified by nodes to see that all transactions in that block are valid before it can be accepted and placed on the chain. A block containing invalid transactions will then be discarded by the nodes. Lets say nodes broadcast different variations of a block at the same time
\subsection{Security}
The Bitcoin Blockchain is secured using cryptographic algorithms and protocols such as hashing and asymmetric private/public keys. For instance, a private key is used to create a digital signature for each transaction a user broadcasts to the network. The digital signature is used to authenticate that the transaction belongs to the user and also prevents data corruption and manipulation from attackers. The private key is anonymous while the public key is used as proof that the transaction belongs to the particular user. 
\subsection{Distributed Data Storage}
\subsection{Decentralization}
\section{Conclusion}

\begin{thebibliography}{1}

\bibitem{Bitcoin}
Satoshi Nakamoto, \emph{Bitcoin: A Peer-to-Peer Electronic Cash System}, https://bitcoin.org/bitcoin.pdf, 2008.
\bibitem{blockchain}
https://cacm.acm.org/magazines/2016/11/209132-blockchain-beyond-bitcoin/fulltext
\bibitem{}
Decentralized Blockchain Technology and Rise of Lex Cryptographia
\bibitem{}
Blockchain: Blueprint for a new economy
\end{thebibliography}

\end{document}
